% Options for packages loaded elsewhere
\PassOptionsToPackage{unicode}{hyperref}
\PassOptionsToPackage{hyphens}{url}
%
\documentclass[
]{article}
\usepackage{amsmath,amssymb}
\usepackage{iftex}
\ifPDFTeX
  \usepackage[T1]{fontenc}
  \usepackage[utf8]{inputenc}
  \usepackage{textcomp} % provide euro and other symbols
\else % if luatex or xetex
  \usepackage{unicode-math} % this also loads fontspec
  \defaultfontfeatures{Scale=MatchLowercase}
  \defaultfontfeatures[\rmfamily]{Ligatures=TeX,Scale=1}
\fi
\usepackage{lmodern}
\ifPDFTeX\else
  % xetex/luatex font selection
\fi
% Use upquote if available, for straight quotes in verbatim environments
\IfFileExists{upquote.sty}{\usepackage{upquote}}{}
\IfFileExists{microtype.sty}{% use microtype if available
  \usepackage[]{microtype}
  \UseMicrotypeSet[protrusion]{basicmath} % disable protrusion for tt fonts
}{}
\makeatletter
\@ifundefined{KOMAClassName}{% if non-KOMA class
  \IfFileExists{parskip.sty}{%
    \usepackage{parskip}
  }{% else
    \setlength{\parindent}{0pt}
    \setlength{\parskip}{6pt plus 2pt minus 1pt}}
}{% if KOMA class
  \KOMAoptions{parskip=half}}
\makeatother
\usepackage{xcolor}
\usepackage[margin=1in]{geometry}
\usepackage{color}
\usepackage{fancyvrb}
\newcommand{\VerbBar}{|}
\newcommand{\VERB}{\Verb[commandchars=\\\{\}]}
\DefineVerbatimEnvironment{Highlighting}{Verbatim}{commandchars=\\\{\}}
% Add ',fontsize=\small' for more characters per line
\usepackage{framed}
\definecolor{shadecolor}{RGB}{248,248,248}
\newenvironment{Shaded}{\begin{snugshade}}{\end{snugshade}}
\newcommand{\AlertTok}[1]{\textcolor[rgb]{0.94,0.16,0.16}{#1}}
\newcommand{\AnnotationTok}[1]{\textcolor[rgb]{0.56,0.35,0.01}{\textbf{\textit{#1}}}}
\newcommand{\AttributeTok}[1]{\textcolor[rgb]{0.13,0.29,0.53}{#1}}
\newcommand{\BaseNTok}[1]{\textcolor[rgb]{0.00,0.00,0.81}{#1}}
\newcommand{\BuiltInTok}[1]{#1}
\newcommand{\CharTok}[1]{\textcolor[rgb]{0.31,0.60,0.02}{#1}}
\newcommand{\CommentTok}[1]{\textcolor[rgb]{0.56,0.35,0.01}{\textit{#1}}}
\newcommand{\CommentVarTok}[1]{\textcolor[rgb]{0.56,0.35,0.01}{\textbf{\textit{#1}}}}
\newcommand{\ConstantTok}[1]{\textcolor[rgb]{0.56,0.35,0.01}{#1}}
\newcommand{\ControlFlowTok}[1]{\textcolor[rgb]{0.13,0.29,0.53}{\textbf{#1}}}
\newcommand{\DataTypeTok}[1]{\textcolor[rgb]{0.13,0.29,0.53}{#1}}
\newcommand{\DecValTok}[1]{\textcolor[rgb]{0.00,0.00,0.81}{#1}}
\newcommand{\DocumentationTok}[1]{\textcolor[rgb]{0.56,0.35,0.01}{\textbf{\textit{#1}}}}
\newcommand{\ErrorTok}[1]{\textcolor[rgb]{0.64,0.00,0.00}{\textbf{#1}}}
\newcommand{\ExtensionTok}[1]{#1}
\newcommand{\FloatTok}[1]{\textcolor[rgb]{0.00,0.00,0.81}{#1}}
\newcommand{\FunctionTok}[1]{\textcolor[rgb]{0.13,0.29,0.53}{\textbf{#1}}}
\newcommand{\ImportTok}[1]{#1}
\newcommand{\InformationTok}[1]{\textcolor[rgb]{0.56,0.35,0.01}{\textbf{\textit{#1}}}}
\newcommand{\KeywordTok}[1]{\textcolor[rgb]{0.13,0.29,0.53}{\textbf{#1}}}
\newcommand{\NormalTok}[1]{#1}
\newcommand{\OperatorTok}[1]{\textcolor[rgb]{0.81,0.36,0.00}{\textbf{#1}}}
\newcommand{\OtherTok}[1]{\textcolor[rgb]{0.56,0.35,0.01}{#1}}
\newcommand{\PreprocessorTok}[1]{\textcolor[rgb]{0.56,0.35,0.01}{\textit{#1}}}
\newcommand{\RegionMarkerTok}[1]{#1}
\newcommand{\SpecialCharTok}[1]{\textcolor[rgb]{0.81,0.36,0.00}{\textbf{#1}}}
\newcommand{\SpecialStringTok}[1]{\textcolor[rgb]{0.31,0.60,0.02}{#1}}
\newcommand{\StringTok}[1]{\textcolor[rgb]{0.31,0.60,0.02}{#1}}
\newcommand{\VariableTok}[1]{\textcolor[rgb]{0.00,0.00,0.00}{#1}}
\newcommand{\VerbatimStringTok}[1]{\textcolor[rgb]{0.31,0.60,0.02}{#1}}
\newcommand{\WarningTok}[1]{\textcolor[rgb]{0.56,0.35,0.01}{\textbf{\textit{#1}}}}
\usepackage{longtable,booktabs,array}
\usepackage{calc} % for calculating minipage widths
% Correct order of tables after \paragraph or \subparagraph
\usepackage{etoolbox}
\makeatletter
\patchcmd\longtable{\par}{\if@noskipsec\mbox{}\fi\par}{}{}
\makeatother
% Allow footnotes in longtable head/foot
\IfFileExists{footnotehyper.sty}{\usepackage{footnotehyper}}{\usepackage{footnote}}
\makesavenoteenv{longtable}
\usepackage{graphicx}
\makeatletter
\def\maxwidth{\ifdim\Gin@nat@width>\linewidth\linewidth\else\Gin@nat@width\fi}
\def\maxheight{\ifdim\Gin@nat@height>\textheight\textheight\else\Gin@nat@height\fi}
\makeatother
% Scale images if necessary, so that they will not overflow the page
% margins by default, and it is still possible to overwrite the defaults
% using explicit options in \includegraphics[width, height, ...]{}
\setkeys{Gin}{width=\maxwidth,height=\maxheight,keepaspectratio}
% Set default figure placement to htbp
\makeatletter
\def\fps@figure{htbp}
\makeatother
\setlength{\emergencystretch}{3em} % prevent overfull lines
\providecommand{\tightlist}{%
  \setlength{\itemsep}{0pt}\setlength{\parskip}{0pt}}
\setcounter{secnumdepth}{-\maxdimen} % remove section numbering
\ifLuaTeX
  \usepackage{selnolig}  % disable illegal ligatures
\fi
\usepackage{bookmark}
\IfFileExists{xurl.sty}{\usepackage{xurl}}{} % add URL line breaks if available
\urlstyle{same}
\hypersetup{
  pdftitle={Hospital Length of Stays},
  pdfauthor={Keith Douglas},
  hidelinks,
  pdfcreator={LaTeX via pandoc}}

\title{Hospital Length of Stays}
\author{Keith Douglas}
\date{2024-10-15}

\begin{document}
\maketitle

\begin{Shaded}
\begin{Highlighting}[]
\FunctionTok{library}\NormalTok{(tidyverse)}
\FunctionTok{library}\NormalTok{(NHSRdatasets)}
\FunctionTok{library}\NormalTok{(knitr)}
\end{Highlighting}
\end{Shaded}

\section{Load the data from the
package}\label{load-the-data-from-the-package}

\begin{Shaded}
\begin{Highlighting}[]
\CommentTok{\# data("LOS\_model")}
\CommentTok{\# ?LOS\_model}
\end{Highlighting}
\end{Shaded}

\section{Inspect}\label{inspect}

\begin{Shaded}
\begin{Highlighting}[]
\FunctionTok{str}\NormalTok{(LOS\_model)}
\end{Highlighting}
\end{Shaded}

\begin{verbatim}
## tibble [300 x 5] (S3: tbl_df/tbl/data.frame)
##  $ ID          : int [1:300] 1 2 3 4 5 6 7 8 9 10 ...
##  $ Organisation: Ord.factor w/ 10 levels "Trust1"<"Trust2"<..: 1 2 3 4 5 6 7 8 9 10 ...
##  $ Age         : int [1:300] 55 27 93 45 70 60 25 48 51 81 ...
##  $ LOS         : int [1:300] 2 1 12 3 11 7 4 4 7 1 ...
##  $ Death       : int [1:300] 0 0 0 1 0 0 0 0 1 0 ...
\end{verbatim}

\section{Make Death a factor}\label{make-death-a-factor}

\begin{Shaded}
\begin{Highlighting}[]
\NormalTok{hospital\_data }\OtherTok{\textless{}{-}}\NormalTok{ LOS\_model }\SpecialCharTok{\%\textgreater{}\%} 
                  \FunctionTok{mutate}\NormalTok{(}\AttributeTok{Death =} \FunctionTok{factor}\NormalTok{(Death))}
\end{Highlighting}
\end{Shaded}

\section{Recode Death levels}\label{recode-death-levels}

\begin{Shaded}
\begin{Highlighting}[]
\NormalTok{hospital\_data }\OtherTok{\textless{}{-}}\NormalTok{ hospital\_data }\SpecialCharTok{\%\textgreater{}\%} 
  \FunctionTok{mutate}\NormalTok{(}\AttributeTok{Death =}\NormalTok{ Death }\SpecialCharTok{\%\textgreater{}\%} 
           \FunctionTok{fct\_recode}\NormalTok{(}\StringTok{"Survived"} \OtherTok{=} \StringTok{"0"}\NormalTok{, }\StringTok{"Died"} \OtherTok{=} \StringTok{"1"}\NormalTok{))}
\FunctionTok{head}\NormalTok{(hospital\_data)}
\end{Highlighting}
\end{Shaded}

\begin{verbatim}
## # A tibble: 6 x 5
##      ID Organisation   Age   LOS Death   
##   <int> <ord>        <int> <int> <fct>   
## 1     1 Trust1          55     2 Survived
## 2     2 Trust2          27     1 Survived
## 3     3 Trust3          93    12 Survived
## 4     4 Trust4          45     3 Died    
## 5     5 Trust5          70    11 Survived
## 6     6 Trust6          60     7 Survived
\end{verbatim}

\section{Create a summary table where each combination of Organisation
and Death gets a count
(n).}\label{create-a-summary-table-where-each-combination-of-organisation-and-death-gets-a-count-n.}

\begin{Shaded}
\begin{Highlighting}[]
\NormalTok{hospital\_data\_summary }\OtherTok{\textless{}{-}}\NormalTok{ hospital\_data }\SpecialCharTok{\%\textgreater{}\%} 
  \FunctionTok{group\_by}\NormalTok{(Organisation, Death) }\SpecialCharTok{\%\textgreater{}\%} 
  \FunctionTok{tally}\NormalTok{()}
\end{Highlighting}
\end{Shaded}

\section{Make a wide table with Dead and Survived as rows with a column
for each
Trust}\label{make-a-wide-table-with-dead-and-survived-as-rows-with-a-column-for-each-trust}

\begin{Shaded}
\begin{Highlighting}[]
\NormalTok{hospital\_data\_wide }\OtherTok{\textless{}{-}}\NormalTok{ hospital\_data\_summary }\SpecialCharTok{\%\textgreater{}\%} 
  \FunctionTok{pivot\_wider}\NormalTok{(}
    \AttributeTok{names\_from =}\NormalTok{ Organisation,}
    \AttributeTok{values\_from =}\NormalTok{ n}
\NormalTok{  )}
\end{Highlighting}
\end{Shaded}

\section{Another pivot with Survived and Died as columns, Trusts as
rows.}\label{another-pivot-with-survived-and-died-as-columns-trusts-as-rows.}

\section{Also calculate the \% survived for each
Trust}\label{also-calculate-the-survived-for-each-trust}

\begin{Shaded}
\begin{Highlighting}[]
\NormalTok{hospital\_data\_wide\_pretty }\OtherTok{\textless{}{-}}\NormalTok{ hospital\_data\_summary }\SpecialCharTok{\%\textgreater{}\%} 
  \FunctionTok{pivot\_wider}\NormalTok{(}
    \AttributeTok{names\_from =}\NormalTok{ Death,}
    \AttributeTok{values\_from =}\NormalTok{ n}
\NormalTok{  ) }\SpecialCharTok{\%\textgreater{}\%} 
  \FunctionTok{mutate}\NormalTok{(}\AttributeTok{Total =}\NormalTok{ Survived }\SpecialCharTok{+}\NormalTok{ Died,}
         \AttributeTok{Percent\_Survived =}\NormalTok{ (Survived}\SpecialCharTok{/}\NormalTok{Total)}\SpecialCharTok{*}\DecValTok{100}\NormalTok{)}
\end{Highlighting}
\end{Shaded}

\section{Make the wide table pretty with
kable()}\label{make-the-wide-table-pretty-with-kable}

\begin{Shaded}
\begin{Highlighting}[]
\NormalTok{hospital\_data\_wide\_pretty }\SpecialCharTok{\%\textgreater{}\%} 
  \FunctionTok{kable}\NormalTok{(}
    \AttributeTok{col.names =} \FunctionTok{c}\NormalTok{(}\StringTok{"Trust"}\NormalTok{, }\StringTok{"Survived"}\NormalTok{, }\StringTok{"Died"}\NormalTok{, }\StringTok{"Total"}\NormalTok{, }\StringTok{"Percent Survived"}\NormalTok{),}
    \AttributeTok{digits =} \DecValTok{0}\NormalTok{,}
    \AttributeTok{caption =} \StringTok{"Hospital Length of Stays data: Percent Survived by Trust"}\NormalTok{,}
    \AttributeTok{align =} \StringTok{"lcccc"}
\NormalTok{  )}
\end{Highlighting}
\end{Shaded}

\begin{longtable}[]{@{}lcccc@{}}
\caption{Hospital Length of Stays data: Percent Survived by
Trust}\tabularnewline
\toprule\noalign{}
Trust & Survived & Died & Total & Percent Survived \\
\midrule\noalign{}
\endfirsthead
\toprule\noalign{}
Trust & Survived & Died & Total & Percent Survived \\
\midrule\noalign{}
\endhead
\bottomrule\noalign{}
\endlastfoot
Trust1 & 23 & 7 & 30 & 77 \\
Trust2 & 25 & 5 & 30 & 83 \\
Trust3 & 24 & 6 & 30 & 80 \\
Trust4 & 26 & 4 & 30 & 87 \\
Trust5 & 23 & 7 & 30 & 77 \\
Trust6 & 26 & 4 & 30 & 87 \\
Trust7 & 22 & 8 & 30 & 73 \\
Trust8 & 25 & 5 & 30 & 83 \\
Trust9 & 27 & 3 & 30 & 90 \\
Trust10 & 26 & 4 & 30 & 87 \\
\end{longtable}

\begin{Shaded}
\begin{Highlighting}[]
\CommentTok{\#To make pdf knit from dropdown, also make sure unused code is scribbed out and }
\end{Highlighting}
\end{Shaded}

\section{Let's knit to PDF}\label{lets-knit-to-pdf}

\end{document}
